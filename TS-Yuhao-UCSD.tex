%%%%%%%%%%%%%%%%%%%%%%%%%%%%%%%%%%%%%%%%%%%%%%%%%%%%%%%%%%%%%%%%%%%%%
%% Title: SOP LaTeX Template
%% Author: Soonho Kong / soonhok@cs.cmu.edu
%% Created: 2012-11-12
%%%%%%%%%%%%%%%%%%%%%%%%%%%%%%%%%%%%%%%%%%%%%%%%%%%%%%%%%%%%%%%%%%%%%

%%%%%%%%%%%%%%%%%%%%%%%%%%%%%%%%%%%%%%%%%%%%%%%%%%%%%%%%%%%%%%%%%%%%%
%%
%% Requirement:
%%     You need to have the `Adobe Caslon Pro` font family.
%%     For more information, please visit:
%%     http://store1.adobe.com/cfusion/store/html/index.cfm?store=OLS-US&event=displayFontPackage&code=1712
%%
%% How to Compile:
%%     $ xelatex main.tex
%%
%%%%%%%%%%%%%%%%%%%%%%%%%%%%%%%%%%%%%%%%%%%%%%%%%%%%%%%%%%%%%%%%%%%%%

\documentclass[letterpaper]{article}
\usepackage[letterpaper,margin=.9in,noheadfoot]{geometry}
\usepackage{fontspec, color, enumerate, sectsty}
\usepackage[normalem]{ulem}

%%%%%%%%%%%%%%%%%%%%%%%%%%%%%%%%%%%%%%%%%%%%%%%%%%%%%%%%%%%%%%%%%%%%%
%                      YOUR INFORMATION
%
%      PLEASE EDIT THE FOLLOWING LINES ACCORDINGLY!!
%%%%%%%%%%%%%%%%%%%%%%%%%%%%%%%%%%%%%%%%%%%%%%%%%%%%%%%%%%%%%%%%%%%%%
\newcommand{\soptitle}{Teaching Statement}
\newcommand{\yourname}{Yuhao Zhang}
\newcommand{\youremail}{yuz870@eng.ucsd.edu}
\setlength{\footskip}{30pt}

%% FONTS SETUP
\defaultfontfeatures{Mapping=tex-text}
\setromanfont{Adobe Caslon Pro}
\setmonofont[Path = /System/Library/Fonts/, Scale=1]{Monaco}
\setsansfont[Scale=1]{Optima}
\newcommand{\amper}{{\fontspec[Scale=1]{Adobe Caslon Pro}\selectfont\itshape\&~{}}}
\usepackage[bookmarks, colorlinks, breaklinks,
pdftitle={\yourname - \soptitle},pdfauthor={\yourname}, unicode]{hyperref}
\hypersetup{linkcolor=magneta,citecolor=magenta,filecolor=magenta,urlcolor=[named]{WildStrawberry}}

\pagenumbering{gobble}

%%%%%%%%%%%%%%%%%%%%%%%%%%%%%%%%%%%%%%%%%%%%%%%%%%%%%%%%%%%%%%%%%%%%%
%                      Title and Author Name
%%%%%%%%%%%%%%%%%%%%%%%%%%%%%%%%%%%%%%%%%%%%%%%%%%%%%%%%%%%%%%%%%%%%%
\begin{document}
\begin{center}{\huge \scshape \soptitle}\end{center}
\begin{center}\vspace{0.2em} {\Large \yourname\\}
  {\youremail\\} {https://yhzhang.info}\end{center}

%%%%%%%%%%%%%%%%%%%%%%%%%%%%%%%%%%%%%%%%%%%%%%%%%%%%%%%%%%%%%%%%%%%%%
%                      SOP Body
% NOTE: Use \amper instead of \&
%%%%%%%%%%%%%%%%%%%%%%%%%%%%%%%%%%%%%%%%%%%%%%%%%%%%%%%%%%%%%%%%%%%%%
%\section*{Introduction}
\noindent I often imagine about myself standing behind the lectern, giving my first lecture to the first-year students. I am going to tell them that GPA does not matter, but the knowledge and lessons they get will benefit their future career. I will teach them that it is only through scientific research that humanity can strive to live another day, and please contact me if they want to work on a project -- things I wish my college professor had said years ago. I dreamed of seeing my own students succeed, and I shall rejoice. Teaching and mentoring have a sacred place in my heart, and I aspire to have more opportunities to work with students during this postdoc.

\paragraph{My approach and philosophy.} An ancient Chinese proverb can summarize my first principle of teaching: \textit{give a person a fish and you feed them for a day; teach them to fish and you feed them for a lifetime}. The Internet, along with online course videos and textbooks, is a better source of knowledge and facts than any lecturer. The challenging part is to teach the students how to navigate the web of CS knowledge and reach the piece of information they need to solve a challenge. I believe the most important quest of a CS course, instead of making the students memorize an overwhelming amount of knowledge, is to teach the students how to build a map with which they can seek knowledge by themselves in the future. CS is a vast field with numerous exciting and challenging problems, and it is unrealistic to cover even just one subfield fully in only a semester. I believe a professor must guide the students to learn to analyze, categorize, summarize, and deduct a seemingly complex problem into the familiar form they have seen and can readily search from the Internet or textbooks. Take teaching in the SysML field as an example. It is about the understanding of ML algorithms, parallel systems, and how ML workloads are parallelized. I will start with the fundamental OS and parallel system concepts, then move on to how ML workloads differ from the traditional ones, and finally introduce the latest technologies of ML systems, revealing the connections between OS, data systems, and ML systems. This way, when the next generation of ML systems arrives, the students can still follow the same road map and be able to draw connections back to the classical problems themselves. 

My second teaching doctrine is the \textit{emphasis on course projects}. I prefer to guide the students toward hands-on projects, whether engineering- or research-oriented. I believe that CS, as a field, prospered and changed the world primarily because of the constant trying out and implementation of ideas. Through challenging and practical projects, the students can get first-hand experiences and are constantly challenged to acquire the knowledge necessary to solve the puzzles. Course projects also tend to have a long-lasting benefit for the student's future career, whether as engineers who need practical experiences or seeking academic positions that demand research projects.

\paragraph{Teaching and mentoring experiences.} My experiences are primarily during my PhD study at UCSD. I was a teaching assistant for DSC102: Systems for Scalable Analytics, an undergrad course that is first of its kind about large-scale data analytics systems for Data Science majors. I designed the first edition of course assignments involving cloud computing, large-scale data analytics, and machine learning. I coded the programming assignments and wrote guidance and documentation. These materials have since been used in another three offerings and by over 500 students. I also held office hours, answered questions online, and gave talks about PAs and scalable data systems. I also TAed CSE234: Data Systems for Machine Learning, a heavily research-oriented graduate course. I helped 12 MS students with their course projects ranging from advanced implementation of cutting-edge research to evaluation and surveying the state-of-art work and open-ended research. They all gave talks open to the entire campus, and out of which, Vignesh Nanda Kumar and Pradyumna Sridhara are continuing working with me towards a research paper, Abhishek Gupta and Rishikesh Ingale built extensions to my past research project and wrote a tech report. Tanay Karve et al. built a UCSD campus-wide data-sharing platform called Data Planet. Apart from these experiences, I also try to help my fellow PhD students. I regularly provided feedback for oral exams, papers, and talks to Supun Nakandala, Vraj Shah, Xiuwen Zheng, Kabir Nagrecha, and Kyle Luoma.

\paragraph{Courses I want to teach.} I am qualified and would be privileged to teach in the SysML, data mining, and scalable data systems field. I want to teach cutting-edge research-oriented courses at the graduate level and hands-on system courses at the undergraduate level. Apart from the above, I am also happy to teach regular database systems and fundamental algorithm/data structure courses.



%\bibliographystyle{abbrv}
%\bibliography{main}

\end{document}